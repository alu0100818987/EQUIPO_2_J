%%%%%%%%%%%%%%%%%%%%%%%%%%%%%%%%%%%%%%%%%%%%%%%%%%%%%%%%%%%%%%%%%%%%%%%%%%%%%%%
% Chapter 3: Procedimiento experimental 
%%%%%%%%%%%%%%%%%%%%%%%%%%%%%%%%%%%%%%%%%%%%%%%%%%%%%%%%%%%%%%%%%%%%%%%%%%%%%%%


%++++++++++++++++++++++++++++++++++++++++++++++++++++++++++++++++++++++++++++++
\section{Descripci�n de los experimentos}
\label{3:sec:1}
A continuaci�n expondremos los pasos que se han seguido en la elaboraci�n del experimento desarrollado para este trabajo de investigaci�n. 
Nos apoyaremos en gr�ficos y tablas que les ayudaran a reforzar y aclarar la informaci�n desarrollada.
\subsection{Descripci�n de los experimentos}
Para llevar a cabo el m�todo de {\em Taylor}, objetivo principal del informe, se ha empleado la ecuaci�n del m�todo de {\em Taylor}. Recodar que  la funci�n derivada ha sido $f(x)=arcsin(x)$  y que al aplicar la serie de Taylor hemos tenido que calcular hasta la derivada n-�sima y, adem�s, elos factoriales de 1 hasta n.

En relaci�n a la eficiencia del proyecto, se ha analizado el resultado obtenido de la aproximaci�n de Taylor midiendo el error de este con el resultado original de la funci�n. 

%++++++++++++++++++++++++++++++++++++++++++++++++++++++++++++++++++++++++++++++
\section{Descripci�n del material}
\label{3:sec:2}
El material utilizado ha sido el siguiente:
\begin {itemize}
\item \underline{Tipo de CPU}: 
 
 
 Intel(R) Core(TM) i3-2328M CPU @ 2.20GHz 

\item \underline{Tama�o de la memoria del procesador}: 


 3072 KB

\item \underline{Vendedor GenuineIntel}:


Linux

\item \underline{Sistema operativo}:


 66-Ubuntu SMP

\item \underline{Plataforma}:


 Linux-3.2.0-59-generic-pae-i686-with-Ubuntu-12.04-precise

\item \underline{Version}:

2.7.3
\end{itemize}


%++++++++++++++++++++++++++++++++++++++++++++++++++++++++++++++++++++++++++++++

\section{Resultados obtenidos}
\label{3:sec:3}
Como resultados de haber  calculado en el algoritmo para cualquiero polinomio obetemos el valor del polinomio con el m�todo de Taylor, el error y el tiempo\footnote{En segundos}  que tarda el algoritmo en calcularlos.

\newpage
Como ejemplos de los tiempos y errores en diferentes puntos de un polinomio de grado 3, obtenemos representado en la tabla anterior.

%--------------------------------------------------------------------------
\begin{table}{}
 \begin{center}
  \begin{tabular}{|c|c|c|c|}
   \hline
   X     &a& tiempo            & Error \\ \hline
   -1    &1&0.0295159816742    & 4.51842232203082e+220 \\ \hline
   -0.25 &1&0.0272219181061    & 1.76500872128679e+220 \\ \hline
   -0.2  &1&0.0301878452301    & 1.62663203764503e+220 \\ \hline     
   0     &1&0.0254921913147    & 1.12960558199550e+220 \\ \hline
   0.2   &1&0.0249121189117    & 7.22947572667555e+219 \\ \hline
   0.25  &1&0.0299370288849    & 6.35403140081687e+219 \\ \hline
   1     &1&0.0301327705383    & 0 \\ \hline
\end{tabular}
\end{center}
\caption{Experimentos en el algoritmo}
\label{tab}
\end{table}
%------------------------------------------------------------------------------
Por cada dato que calculamos obtemos la funci�n representada en una gr�fica en dicho punto, como por ejemplo esta gr�fica:
%------------------------------------------------------------------------------
\begin{figure}[!th]
\begin{center}
\includegraphics[width=0.75\textwidth]{images/grafica.eps}
\caption{Gr�fica de la funci�n arcsin(x)}
\label{fig:1}
\end{center}
\end{figure}
%------------------------------------------------------------------------------
%++++++++++++++++++++++++++++++++++++++++++++++++++++++++++++++++++++++++++++++
\section{An�lisis de los resultados}
\label{3:sec:4}
Obsevando detenidamente los resultados obetenidos en la tabla anterior, podemos apreciar que cuando m�s se parecen el valor del centro(a) y el puento X de la funci�n, disminuye el valor del error entre el polinomio y la aproximaci�n del polinomio. Por otro lado, cuantos m�s dispares son dichos valores mayor es el error entre el polinomio y el polinomio de aproximaci�n.  

Esto se da en los ejemplos propuestos en la tabla, en donde utilizamos un polinomio de grado 3 para todos los casos. Observamos que en primer caso, el valor del centro es 1 y el de la x es -1, el caso m�s dispar que puede darse, se obtiene el error m�s alto. Mientras, en el �ltimo caso, donde el valor del centro es 1 y de x es 1 se obtiene un valor 0 del error ya que el valor del centro y el punto X son iguales.