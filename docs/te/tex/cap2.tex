%%%%%%%%%%%%%%%%%%%%%%%%%%%%%%%%%%%%%%%%%%%%%%%%%%%%%%%%%%%%%%%%%%%%%%%%%%%%%%%
% Chapter 2: Fundamentos Te�ricos 
%%%%%%%%%%%%%%%%%%%%%%%%%%%%%%%%%%%%%%%%%%%%%%%%%%%%%%%%%%%%%%%%%%%%%%%%%%%%%%%

%++++++++++++++++++++++++++++++++++++++++++++++++++++++++++++++++++++++++++++++



%++++++++++++++++++++++++++++++++++++++++++++++++++++++++++++++++++++++++++++++

\section{El por qu� de el m�todo de Taylor}
\label{2:sec:1}
  La funci�n $p(x)=a_0+a_1x+a_2x^2+ ... +a_nx^n$, en la que los coeficientes $a_k$ son constantes, se llama polinomio de grado n.En particular $y=ax+b$ es un polinomio de grado 1, de los m�s sencillos, por lo que calcular su valor en f�cil. Sin embargo, calcular el valor par
a otras funciones como log(x), sen(x), $e^X$, ... es mucho m�s complicado. Por tanto, se utilizan metodos desarrollados por el an�lisis matem�tico, como el m�todo de Taylor.

  
  \section{Segundo apartado del segundo cap�tulo}
\label{2:sec:2}
  Para poder usar este m�todo deben cumplirse dos condiciones:
\begin{itemize}
  \item Sea f(x) una funci�n continua en [a,b]
  \item Sea f(x) derivable en (a,b)
\end{itemize}
Cuando tengamos un polinomio de primer grado $p_1(x)=f'(a)(x-a)$ tendr� el mismo valor que f(x) en el punto x=a. Dando la gr�fica es una recta tangente a la gr�fica de f(x) en el punto a.\par 
Es pol�sible elegir un polinomio de segundo grado, $p_2(x)=f(a)+f'(a)(x-a)+\frac{1}{2}f''(a)(x-a)^2$, tal que en el punto x=a tenga el mismo valor que f(x) y valores tambi�n iguales para su primera y segunda derivadas.Se gr�fica en el punto a se acercar� a la de f(x) m�s que la anterior. Es natural esperar que si contruimos un polinomio que en x=a tenga las mismas n primeras derivadas que f(x) en el mismo punto, este polinomio se aproximar� m�s a f(x) en los puntos x pr�ximos a a. As� obtenemos la siguiente igualdad aproximada, que es la f�rmula de taylor:
$$f(x)\approx f(a)+f'(a)(x-a)+(\frac{1}{2}!)f''(a)(x-a)^2+ ... + (\frac{1}{n}!)f^(n)(a)(x-a)^n$$

Sin embargo, esto solo se da para polinomios que tengan su derivada hasta n, mientras que para los polinomios que
 tienen derivada (n+1)-�sima difieren de f(x) en una peque�a cantidad, que denominamos como el error.\par
 Por ello a�adimos un t�rmino m�s, llamado resto, para que el error sea menor:
 $$f(x)=f(a)+f'(a)(x-a)+(\frac{1}{2})!f''(a)(x-a)^2+...+(\frac{1}{n})!f^(n)(a)(x-a)^n+(\frac{1}{(n+1)!})f^(n+1)(c)(x-a)^(n+1)$$