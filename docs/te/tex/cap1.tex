%%%%%%%%%%%%%%%%%%%%%%%%%%%%%%%%%%%%%%%%%%%%%%%%%%%%%%%%%%%%%%%%%%%%%%%%%%%%%
% Chapter 1: Motivaci�n y Objetivos 
%%%%%%%%%%%%%%%%%%%%%%%%%%%%%%%%%%%%%%%%%%%%%%%%%%%%%%%%%%%%%%%%%%%%%%%%%%%%%%%

%Los objetivos le dan al lector las razones por las que se realiz� el
%proyecto o trabajo de investigaci�n.

%---------------------------------------------------------------------------------
\section{Motivaci�n}
\label{1:sec:1}
Este proyecto o trabajo de investigaci�n se realiz� con la finalidad de crear un programa en \textsf{Python} en el que se calcula por el m�todo de Taylor la funci�n de arcseno(x).Adem�s, proponerlo en un informe de tipo \LaTeX{}~\cite{LaTeXpg}, y posteriormente, una presentaci�n en \textsc{Beamer}.


%---------------------------------------------------------------------------------
\section{Objetivos}
\label{1:sec:2}

Los objetivos de esta practica es aprender a manejar tanto el \textsf{Python}, el \LaTeX{} y el \textsc{Beamer}:\par
\begin{itemize}
  \item \textsf{Python}
  \item \LaTeX{}: Un documento en Latex consiste en un texto propiamente tal y una serie de comandos para el compilador que son los que le van a dar la forma al texto.
  \item  \textsc{Beamer}
\end{itemize}