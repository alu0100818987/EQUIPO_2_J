%%%%%%%%%%%%%%%%%%%%%%%%%%%%%%%%%%%%%%%%%%%%%%%%%%%%%%%%%%%%%%%%%%%%%%%%%%%%%
% Chapter 1: Motivaci�n y Objetivos 
%%%%%%%%%%%%%%%%%%%%%%%%%%%%%%%%%%%%%%%%%%%%%%%%%%%%%%%%%%%%%%%%%%%%%%%%%%%%%%%

%Los objetivos le dan al lector las razones por las que se realiz� el
%proyecto o trabajo de investigaci�n.

%---------------------------------------------------------------------------------
\section{Motivaci�n}
\label{1:sec:1}
Este proyecto o trabajo de investigaci�n se realiz� con la finalidad de crear un programa en \textsf{Python} en el que se calcula por el m�todo de Taylor la funci�n de arcseno(x).Adem�s, proponerlo en un informe de tipo \LaTeX{}~\cite{LaTeXpg}, y posteriormente, una presentaci�n en \textsc{Beamer}.


%---------------------------------------------------------------------------------
\section{Objetivos}
\label{1:sec:2}

Los objetivos de esta practica es aprender a manejar tanto el \textsf{Python}, el \LaTeX{} y el \textsc{Beamer}:\par
\begin{itemize}
  \item \textsf{Python}es un lenguaje de programaci�n interpretado cuya filosof�a hace hincapi� en una sintaxis muy limpia y que favorezca un c�digo legible.

Se trata de un lenguaje de programaci�n multiparadigma, ya que soporta orientaci�n a objetos, programaci�n imperativa y, en menor medida, programaci�n funcional. Es un lenguaje interpretado, usa tipado din�mico y es multiplataforma.
  \item \LaTeX{} es un sistema de composici�n de textos, orientado especialmente a la creaci�n de libros, documentos cient�ficos y t�cnicos que contengan f�rmulas matem�ticas.
  \item  \textsc{Beamer} es una clase de \LaTeX{} para la creaci�n de presentaciones, funciona con pdflatex.
\end{itemize}