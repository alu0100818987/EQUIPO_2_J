%%%%%%%%%%%%%%%%%%%%%%%%%%%%%%%%%%%%%%%%%%%%%%%%%%%%%%%%%%%%%%%%%%%%%%%%%%%%%
% Chapter 1: Motivaci�n y Objetivos 
%%%%%%%%%%%%%%%%%%%%%%%%%%%%%%%%%%%%%%%%%%%%%%%%%%%%%%%%%%%%%%%%%%%%%%%%%%%%%%%

Los objetivos le dan al lector las razones por las que se realiz� el
proyecto o trabajo de investigaci�n.

%---------------------------------------------------------------------------------
\section{Secci�n Uno}
\label{1:sec:1}
 Si simplemente se desea escribir texto normal en LaTeX, 
 sin complicadas f�rmulas matem�ticas o efectos especiales
 como cambios de fuente, entonces simplemente tiene que escribir
 en espa�ol normalmente.
\par
 Si desea cambiar de p�rrafo ha de dejar una l�nea en blanco o bien
 utilizar el comando \par
 No es necesario preocuparse de la sangr�a de los p�rrafos:
 todos los p�rrafos se sangrar�n autom�ticamente con la excepci�n
 del primer p�rrafo de una secci�n.\par
 Se ha de distinguir entre la comilla simple 'izquierda'
 y la comilla simple 'derecha' cuando se escribe en el ordenador.
 En el caso de que se quieran utilizar comillas dobles se han de
 escribir dos caracteres 'comilla simple' seguido, esto es, 
 "comillas dobles".
 Tambi�n se ha de tener cuidado con los guiones: se utiliza un �nico
 gui�n para la separaci�n de s�labas, mientras que se utilizan
 tres guiones seguidos para producir un gui�n de los que se usan
 como signo de puntuaci�n --- como en esta oraci�n.


%---------------------------------------------------------------------------------
\section{Secci�n Dos}
\label{1:sec:2}
  Primer p�rrafo de la segunda secci�n.

\begin{itemize}
  \item Item 1
  \item Item 2
  \item Item 3
\end{itemize}

