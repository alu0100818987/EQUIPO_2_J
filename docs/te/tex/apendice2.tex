\section{Otro apendice: Algoritmo}
\label{Apendice2:label}

\begin{center}
\begin{footnotesize}

\begin{verbatim}
En la primera parte de algoritmo, hemos importado las diferentes librerias que necesitamos.
Luego declaramos dos funciones: El la primera calculamos el factorial, que necesitamos para 
calcular la formula de Taylor, donde podemos ver que solo usamos dos simples condiciones, y 
despues declaramos la funcion Taylor, para calcularla.

En esta segunda funcion, declaramos un variacle ('c') general que usamos cuando evaluamos la
funcion en el centro(a). Para luego usarlo en un bucle "for" donde calcula la deriva de la funcion,
la evalua en el punto centro(a) y opera, invocando a la funcion factorial, obteniendo la aproximacion
de funcion. Por ultimo, se iguala la funcion a la deriva para que así siga el bucle tantas veces
como el valor del grado(n).

Y como resultado se devuelve el resultado de la suma con un "return suma".

  
Entonces en esta segunda parte, se piden los valores del grado(n), el punto(x) y el centro(a), aunque
se vuelven a pedir los valores de x y a si no estan entre el intervalos [-1,1].Depues, se declara una 
variable inicial para empezar a medir el tiempo que tarda el programa en ejecutarse. Se invoca a la funcion
principal, taylor, y se calcula con los valores que se hallan introducido. Y se declara otra variable final 
en la que se obtiene el valor del tiempo que ha tardado.

Luego, se calcula el error con un valor absoluto(para que no de valores negativos) entre el valor exacto 
de la funcion menos la aproximacion de dicha funcion. Finalmente, se imprimen por pantalla todos los valores.
\end{verbatim}

\end{footnotesize}
\end{center}

\section{Otro apendice: Representacion de la funcion}
\label{Apendice2:label2}

\begin{center}
\begin{footnotesize}

\begin{verbatim}
 En la ultima parte del programa, ya que hemos realizado todo lo que nos pedia en el mismo
 algoritmo, representamos la funcion pidiendo valores para ambos ejes e iniciamos una lista
 para guardar los valores que se van tomando del "for". Proponemos las medidas de la grafica y 
 demas detalles, como los colores, la anchura de las lines, ect. Declaramos entre que intervalos
 debe de darse la grafica y la llamamos "Representacion grafica". Por ultimo, la guardamos en
 un fichero. 
\end{verbatim}


\end{footnotesize}
\end{center}
